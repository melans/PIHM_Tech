\documentclass[]{scrbook}
\usepackage{lmodern}
\usepackage{amssymb,amsmath}
\usepackage{ifxetex,ifluatex}
\usepackage{fixltx2e} % provides \textsubscript
\ifnum 0\ifxetex 1\fi\ifluatex 1\fi=0 % if pdftex
  \usepackage[T1]{fontenc}
  \usepackage[utf8]{inputenc}
\else % if luatex or xelatex
  \ifxetex
    \usepackage{mathspec}
  \else
    \usepackage{fontspec}
  \fi
  \defaultfontfeatures{Ligatures=TeX,Scale=MatchLowercase}
\fi
% use upquote if available, for straight quotes in verbatim environments
\IfFileExists{upquote.sty}{\usepackage{upquote}}{}
% use microtype if available
\IfFileExists{microtype.sty}{%
\usepackage[]{microtype}
\UseMicrotypeSet[protrusion]{basicmath} % disable protrusion for tt fonts
}{}
\PassOptionsToPackage{hyphens}{url} % url is loaded by hyperref
\usepackage[unicode=true]{hyperref}
\hypersetup{
            pdftitle={Penn State Integrated Hydrologic Model(PIHM)},
            pdfauthor={Lele Shu},
            pdfborder={0 0 0},
            breaklinks=true}
\urlstyle{same}  % don't use monospace font for urls
\usepackage{natbib}
\bibliographystyle{apalike}
\usepackage{longtable,booktabs}
% Fix footnotes in tables (requires footnote package)
\IfFileExists{footnote.sty}{\usepackage{footnote}\makesavenoteenv{long table}}{}
\usepackage{graphicx,grffile}
\makeatletter
\def\maxwidth{\ifdim\Gin@nat@width>\linewidth\linewidth\else\Gin@nat@width\fi}
\def\maxheight{\ifdim\Gin@nat@height>\textheight\textheight\else\Gin@nat@height\fi}
\makeatother
% Scale images if necessary, so that they will not overflow the page
% margins by default, and it is still possible to overwrite the defaults
% using explicit options in \includegraphics[width, height, ...]{}
\setkeys{Gin}{width=\maxwidth,height=\maxheight,keepaspectratio}
\IfFileExists{parskip.sty}{%
\usepackage{parskip}
}{% else
\setlength{\parindent}{0pt}
\setlength{\parskip}{6pt plus 2pt minus 1pt}
}
\setlength{\emergencystretch}{3em}  % prevent overfull lines
\providecommand{\tightlist}{%
  \setlength{\itemsep}{0pt}\setlength{\parskip}{0pt}}
\setcounter{secnumdepth}{5}
% Redefines (sub)paragraphs to behave more like sections
\ifx\paragraph\undefined\else
\let\oldparagraph\paragraph
\renewcommand{\paragraph}[1]{\oldparagraph{#1}\mbox{}}
\fi
\ifx\subparagraph\undefined\else
\let\oldsubparagraph\subparagraph
\renewcommand{\subparagraph}[1]{\oldsubparagraph{#1}\mbox{}}
\fi

% set default figure placement to htbp
\makeatletter
\def\fps@figure{htbp}
\makeatother

\usepackage{booktabs}

\title{Penn State Integrated Hydrologic Model(PIHM)}
\providecommand{\subtitle}[1]{}
\subtitle{Technical Documentation}
\author{Lele Shu}
\date{2019-05-03}

\begin{document}
\maketitle

{
\setcounter{tocdepth}{1}
\tableofcontents
}
\chapter{Overview}\label{Overview}

This is the technical documentation of the PIHM system (PIHM and
PIHMgisR).

\textbf{PIHM} The Penn State Integrated Hydrologic Model (PIHM) is a
multiprocess, multi-scale hydrologic model where the major hydrological
processes are fully coupled using the semi-discrete Finite Volume
Method.

\textbf{PIHMGIS} The model itself is ``tightly-coupled'' with PIHMgis,
an open-source Geographical Information System designed for PIHM. The
PIHMgis provides access to the digital data sets (terrain, forcing and
parameters) and tools necessary to drive the model, as well as a
collection of GIS-based pre- and post-processing tools.

Collectively the system is referred to as the \textbf{Penn State
Integrated Hydrologic Modeling System (PIHMS)}.

The PIHM is an open source software, freely available for download at
\href{www.pihm.psu.edu}{PIHM website} or
\href{https://github.com/shulele/PIHM++}{Github Page} along with
installation and user guides.

\section{Why PIHM?}\label{why-pihm}

It is our intention to begin a debate on the role of \emph{Community
Models} in the hydrologic sciences. Our research is a response to recent
trends in US funding for \emph{Observatory Science} that have emerged at
NSF over the last few years, namely, the NSF-funded \textbf{CUAHSI}
program (Consortium of Universities for Advancing Hydrologic Sciences).

PIHM represents our strategy for the synthesis of \emph{multi-state},
\emph{multiscale} distributed hydrologic models using the integral
representation of the underlying physical process equations and state
variables.

Our interest is in devising a concise representation of watershed and/or
river basin hydrodynamics, which allows interactions among major
physical processes operating simultaneously, but with the flexibility to
add or eliminate states/processes/constitutive relations depending on
the objective of the numerical experiment or purpose of the scientific
or operational application.

To satisfy the objectives, the PIHM

\begin{itemize}
\tightlist
\item
  is distributed hydrologic model, based on the semi-discrete
  \textbf{Finite Volume Method (FVM)} in which domain discretization is
  an unstructured triangular irregular network (e.g.~Delaunay triangles)
  generated with constraints (geometric, and parametric). A local
  prismatic control volume is formed by the vertical projection of the
  Delaunay triangles forming each layer of the model. Given a set of
  constraints (e.g.~river network support, watershed boundary, altitude
  zones, ecological regions, hydraulic properties, climate zones, etc),
  an ``optimal'' mesh is generated. River volume elements are also
  prismatic, with trapezoidal or rectangular cross-section, and are
  generated along or cross edges of Delaunay triangles. The local
  control volume contains all equations to be solved and is referred to
  as the model kernel.
\item
  is a physically-based model, in which all equations used are
  describing the physics of the hydrological processes which control the
  catchment. The physical model is able to predict the water in the
  ungage water system, to estimate the sediment, pullutants, and
  vegetation, etc, such that it is practical to be coupled with
  biochemistry, geomorphology, limnology, and other water-related
  research. The global ODE system is assembled by combining all local
  ODE systems throughout the domain and then solved by a
  state-of-the-art parallel ODE solver known as CVODE developed at the
  Lawrence Livermore National Laboratory.
\item
  is a fully-coupled hydrologic model, where the state and flux
  variables in the hydrologic system are solved within the same time
  step and conserve the mass. The fluxes are infiltration, overland
  flow, groundwater recharge, lateral groundwater flow, exchange of
  river and soil/groundwater and river discharge.
\item
  is of an adaptable temporal and spatial resolution. The spatial
  resolution of the model varies from meters to kilometers based
  requirement of modeling and computing resources. The internal time
  step of the iteration step is adjustable; it is able to export the
  status of the catchment in less 1 second to days. Also, the time
  interval for exporting results is configured flexibly. The flexible
  spatial and temporal resolution is rather valuable for community model
  coupling.
\item
  is an open source model, anyone can access the source code, use and
  submit their improvement.
\item
  is a long-term yield and single-event flood model.
\end{itemize}

An important partnership and motivation for this work was the Project
Leaders participation in two community-science research activities over
the last few years: The University of Arizona-led Science and Technology
Center (SAHRA: Sustainability of Water Resources in Semi-Arid Regions),
and the Chesapeake Community Modeling Project (CCMP). Each of these
research programs has been essential in supporting the concept of
\textbf{Community Models} for environmental prediction and helping to
make it happen.

\section{History of PIHM system}\label{history-of-pihm-system}

\begin{itemize}
\tightlist
\item
  2005 PIHM v1.0
\end{itemize}

Dr.~Yizhong Qu \citep{Qu2007} developed and verified the first version
of PIHM in 2001-2005 during his Ph.D.~in Pennsylvania State Unversity,
following the blueprint of Freeze and Harlan (1969). This version of
PIHM is the soul of the PIHM model.

\begin{itemize}
\tightlist
\item
  2009 PIHMgis
\end{itemize}

Dr.~Gopal Bhartt \citep{Bhatt2012} developed the PIHMgis with support of
C++, Qt GUI library, TRIANGLE library, and QGIS developing kit. The
development of PIHMgis makes the learning curve of PIHM moderate and
benefits the developing, modeling and coupling.

\begin{itemize}
\tightlist
\item
  2015 MM-PIHM
\end{itemize}

Dr.~Yuninh Shi led and developed the MM-PIHM (Multi-Module PIHM), which
embedded all modules from PIHM family, such as RT-PIHM, LE-PIHM,
flux-PIHM, BGC-PIHM, etc. together. The sophisticated design and
coupling of the MM-PIHM is the summit of the PIHM as a \emph{Community
Model} that combined all water-related modules together.

\begin{itemize}
\tightlist
\item
  2019 PIHM++
\end{itemize}

Based on the accumulated contribution of PIHM modeling and coupling with
related researches, it is necessary to solve the known bugs and
limitations, improve the performance of the model with parallel methods,
and adopt new updates from SUNDIALS solver and programming strategy.

Several publications that may helps:

\begin{itemize}
\tightlist
\item
  \citep{Qu2004}
\item
  \citep{Qu2007}
\item
  \citep{Li2008}
\item
  \citep{Kumar2004a}
\item
  \citep{Kumar2009d}
\item
  \citep{Yu2015}
\item
  \citep{Yu2014}
\item
  \citep{Li2011}
\item
  \citep{Shi2015}
\item
  \citep{Shi2015a}
\item
  \citep{Bhatt2014}
\end{itemize}

\section{Steps of PIHM modeling}\label{steps-of-pihm-modeling}

\subsection{Essential Terrestrial
Variables?}\label{essential-terrestrial-variables}

\begin{itemize}
\tightlist
\item
  Atmospheric forcing (precipitation, snow cover, wind, relative
  humidity, temperature, net radiation, albedo, photosynthetic
  atmospheric radiation, leaf area index)
\item
  Digital elevation model (DEM)
\item
  River/stream discharge
\item
  Soil (class, hydrologic properties)
\item
  Groundwater (levels, extent, hydro-geologic properties)
\item
  Lake/Reservoir (levels, extent)
\item
  Land cover and land use (biomass, human infrastructure, demography,
  ecosystem disturbance)
\item
  Water use
\end{itemize}

Most data reside on federal servers \ldots{}.many petabytes

\subsection{A-Priori Data Sources}\label{a-priori-data-sources}

\begin{longtable}[]{@{}ccc@{}}
\toprule
\begin{minipage}[b]{0.11\columnwidth}\centering\strut
Feature/Time-Series\strut
\end{minipage} & \begin{minipage}[b]{0.19\columnwidth}\centering\strut
Property\strut
\end{minipage} & \begin{minipage}[b]{0.42\columnwidth}\centering\strut
Source\strut
\end{minipage}\tabularnewline
\midrule
\endhead
\begin{minipage}[t]{0.11\columnwidth}\centering\strut
Soil\strut
\end{minipage} & \begin{minipage}[t]{0.19\columnwidth}\centering\strut
Porosity; Sand, Silt, Clay Fractions; Bulk Density\strut
\end{minipage} & \begin{minipage}[t]{0.42\columnwidth}\centering\strut
CONUS, SSURGO and STATSGO\strut
\end{minipage}\tabularnewline
\begin{minipage}[t]{0.11\columnwidth}\centering\strut
Geology\strut
\end{minipage} & \begin{minipage}[t]{0.19\columnwidth}\centering\strut
Bed Rock Depth; Horizontal and Vertical Hydraulic Conductivity\strut
\end{minipage} & \begin{minipage}[t]{0.42\columnwidth}\centering\strut
\url{http://www.dcnr.state.pa.us/topogeo/},
\url{http://www.lias.psu.edu/emsl/guides/X.html}\strut
\end{minipage}\tabularnewline
\begin{minipage}[t]{0.11\columnwidth}\centering\strut
Land Cover\strut
\end{minipage} & \begin{minipage}[t]{0.19\columnwidth}\centering\strut
LAI\strut
\end{minipage} & \begin{minipage}[t]{0.42\columnwidth}\centering\strut
\href{http://glcf.umiacs.umd.edu/data/landcover/data.shtml}{UMC},
\href{http://ldas.gsfc.nasa.gov/LDAS8th/MAPPED.VEG/LDASmapveg.shtml}{LDASmapveg};\strut
\end{minipage}\tabularnewline
\begin{minipage}[t]{0.11\columnwidth}\centering\strut
Land Cover\strut
\end{minipage} & \begin{minipage}[t]{0.19\columnwidth}\centering\strut
Manning's Roughness;\strut
\end{minipage} & \begin{minipage}[t]{0.42\columnwidth}\centering\strut
Hernandez et. al., 2000\strut
\end{minipage}\tabularnewline
\begin{minipage}[t]{0.11\columnwidth}\centering\strut
River\strut
\end{minipage} & \begin{minipage}[t]{0.19\columnwidth}\centering\strut
Manning's Roughness;\strut
\end{minipage} & \begin{minipage}[t]{0.42\columnwidth}\centering\strut
Dingman (2002)\strut
\end{minipage}\tabularnewline
\begin{minipage}[t]{0.11\columnwidth}\centering\strut
River\strut
\end{minipage} & \begin{minipage}[t]{0.19\columnwidth}\centering\strut
Coefficient of Discharge\strut
\end{minipage} & \begin{minipage}[t]{0.42\columnwidth}\centering\strut
ModHms Manual (Panday and Huyakorn, 2004)\strut
\end{minipage}\tabularnewline
\begin{minipage}[t]{0.11\columnwidth}\centering\strut
River\strut
\end{minipage} & \begin{minipage}[t]{0.19\columnwidth}\centering\strut
Shape and Dimensions;\strut
\end{minipage} & \begin{minipage}[t]{0.42\columnwidth}\centering\strut
Derived from regression using depth, width, and discharge data from
\href{http://nwis.waterdata.usgs.gov/usa/nwis/measurements}{USGS
data}\strut
\end{minipage}\tabularnewline
\begin{minipage}[t]{0.11\columnwidth}\centering\strut
River\strut
\end{minipage} & \begin{minipage}[t]{0.19\columnwidth}\centering\strut
Topology: Nodes, Neighboring Elements;\strut
\end{minipage} & \begin{minipage}[t]{0.42\columnwidth}\centering\strut
Derived using PIHMgis (Bhatt et. al., 2008)\strut
\end{minipage}\tabularnewline
\begin{minipage}[t]{0.11\columnwidth}\centering\strut
Forcing\strut
\end{minipage} & \begin{minipage}[t]{0.19\columnwidth}\centering\strut
Prec, Temp. RH, Wind, Rad.\strut
\end{minipage} & \begin{minipage}[t]{0.42\columnwidth}\centering\strut
National Land Data Assimilation System: NLDAS-2\strut
\end{minipage}\tabularnewline
\begin{minipage}[t]{0.11\columnwidth}\centering\strut
Topography\strut
\end{minipage} & \begin{minipage}[t]{0.19\columnwidth}\centering\strut
DEM\strut
\end{minipage} & \begin{minipage}[t]{0.42\columnwidth}\centering\strut
\url{http://seamless.usgs.gov/}\strut
\end{minipage}\tabularnewline
\begin{minipage}[t]{0.11\columnwidth}\centering\strut
Streamflow\strut
\end{minipage} & \begin{minipage}[t]{0.19\columnwidth}\centering\strut
\strut
\end{minipage} & \begin{minipage}[t]{0.42\columnwidth}\centering\strut
\url{http://nwis.waterdata.usgs.gov/nwis/sw}\strut
\end{minipage}\tabularnewline
\begin{minipage}[t]{0.11\columnwidth}\centering\strut
Groundwater\strut
\end{minipage} & \begin{minipage}[t]{0.19\columnwidth}\centering\strut
\strut
\end{minipage} & \begin{minipage}[t]{0.42\columnwidth}\centering\strut
\url{http://nwis.waterdata.usgs.gov/nwis/gw}\strut
\end{minipage}\tabularnewline
\bottomrule
\end{longtable}

\chapter{Workflow of PIHM System}\label{workflow-of-pihm-system}

\begin{figure}
\centering
\includegraphics{./Fig/Workflow.png}
\caption{The workflow of modeling with PIHM System}
\end{figure}

\begin{enumerate}
\def\labelenumi{\arabic{enumi}.}
\tightlist
\item
  Prepare raw Essential Terrestrial Variables (ETV)
\item
  Convert and crop raw data with the research area boundary.
\item
  Build the PIHM modeling domain with PIHMgis or PIHMgisR (recommended
  for PIHM++)
\item
  Run PIHM on desktop or cluster.
\item
  Analysis the PIHM results with PIHMgisR.
\end{enumerate}

\chapter{Install PIHM and PIHMgisR}\label{install-pihm-and-pihmgisr}

\section{SUNDIALS/CVODE}\label{sundialscvode}

The PIHM model requires the support of SUNDIALS or CVODE library.
\href{https://computation.llnl.gov/projects/sundials}{\textbf{SUNDIALS}}
is a SUite of Nonlinear and DIfferential/ALgebraic equation Solvers,
consists of six solvers.
\href{https://computation.llnl.gov/projects/sundials/cvode}{\textbf{CVODE}}
is a solver for stiff and nonstiff ordinary differential equation (ODE)
systems (initial value problem) given in explicit form \(y' = f(t,y)\).
The methods used in CVODE are variable-order, variable-step multistep
methods. You can install the entire SUNDIALS suite or CVODE only.

Since the SUNDIALS/CVODE keeps updating periodically and significantly,
the function names and structure are changed accordingly, we suggest to
use the specific version of the solver, rather than the latest solver.

\begin{longtable}[]{@{}cc@{}}
\toprule
PIHM Version & SUNDIALS/CVODE version\tabularnewline
\midrule
\endhead
PIHM v1.x & v2.2 \textasciitilde{} v2.4\tabularnewline
PIHM v2.x & v2.2 \textasciitilde{} v2.4\tabularnewline
PIHM v3.x & v2.2 \textasciitilde{} v2.4\tabularnewline
MM-PIHM v1.x & v2.4\tabularnewline
PIHM++ v4.x & v3.x\tabularnewline
\bottomrule
\end{longtable}

SUNDIALS/CVODE is available in
\href{https://computation.llnl.gov/projects/sundials/sundials-software}{LLNL:
https://computation.llnl.gov/projects/sundials/sundials-software}

The installation of CVODE v3.x:

\begin{enumerate}
\def\labelenumi{\arabic{enumi}.}
\item
  Go to your Command Line and enter your workspace and unzip your CVODE
  source code here.
\item
  make directories for CVODE, including \emph{builddir}, \emph{instdir}
  and \emph{srcdir}

\begin{verbatim}
mkdir builddir
mkdir instdir
mkdir srcdir
cd builddir/
\end{verbatim}
\item
  Try ccmake. Install \texttt{cmake} if you don't have one.

\begin{verbatim}
ccmake 
\end{verbatim}
\item
  Run ccmake to configure your compile environment.

\begin{verbatim}
ccmake /Users/leleshu/Dropbox/PIHM/sundials/cvode-4.1.0
\end{verbatim}

  \includegraphics{Fig/ccmake/1.png} This is an empty configure. Press
  \texttt{c} to start the configuration.
\end{enumerate}

\includegraphics{Fig/ccmake/2.png} The default configuration. Make sure
the value for three lines:

\begin{verbatim}
BUILD_CVODE = ON
CMAKE_INSTALL_PREFIX = /usr/local/sundials
EXAMPLES_INSTALL_PATH = /usr/local/sundials/examples
\end{verbatim}

After the modification of values, press \texttt{c} to confirm
configuration.

\includegraphics{Fig/ccmake/3.png} The ccmake configures the environment
automatically. When the configuration is ready, press \texttt{g} to
generate and exit.

\begin{enumerate}
\def\labelenumi{\arabic{enumi}.}
\item
  Then you run commands below:

\begin{verbatim}
make
make install 
\end{verbatim}
\item
  Optional library copy Sometimes, the code might not find the right
  library support in your system, try to copy the library in sundials
  folder to your system library folder.
\end{enumerate}

\begin{verbatim}
cp /usr/local/sundials/lib/* /usr/local/lib/
\end{verbatim}

\section{PIHM}\label{pihm}

Configuration in \emph{Makefile}:

\begin{enumerate}
\def\labelenumi{\arabic{enumi}.}
\tightlist
\item
  Path of \emph{SUNDIALS\_DIR.} {[}\textbf{CRITICAL}{]}
\item
  Path of OpenMP if the parallel is preferred.
\item
  Path of SRC\_DIR, default is \texttt{SRC\_DIR\ =\ .}
\item
  Path of BUILT\_DIR, default is \texttt{BUILT\_DIR\ =\ .}
\end{enumerate}

After updating the SUNDIALS path in the \emph{Makefile}, user can
compile the PIHM with:

\begin{verbatim}
make clean
make pihm
\end{verbatim}

There are more options to compile the PIHM code:

\begin{itemize}
\tightlist
\item
  \texttt{make\ all} - make both pihm and pihm\_omp
\item
  \texttt{make\ pihm} - make pihm executable
\item
  \texttt{make\ pihm\_omp} - make pihm\_omp with OpenMP support
\item
  \texttt{make\ calib\_mpi} - make calib\_mpi with MPI support
\item
  \texttt{make\ calib\_omp} - make calib\_omp with OpenMP support
\end{itemize}

\subsection{OpenMP}\label{openmp}

If parallel-computing is prefered, please install OpenMP. For mac:

\begin{verbatim}
brew install llvm clang
brew install libomp
compile flags for OpenMP: 
  -Xpreprocessor -fopenmp -lomp
Library/Include paths:
  -L/usr/local/opt/libomp/lib 
  -I/usr/local/opt/libomp/include
\end{verbatim}

\subsection{Run pihm executables.}\label{run-pihm-executables.}

After the successful installation and compile, you can run PIHM models
using

\begin{verbatim}
./pihm <projectname>
\end{verbatim}

\includegraphics{Fig/CLI.png} Command line pattern is:

\begin{verbatim}
./pihm [-p projectfile] [-o output_folder] [-n Num_Threads] <project name> 
\end{verbatim}

\begin{itemize}
\tightlist
\item
  \texttt{\textless{}project\ name\textgreater{}} is the name of the
  project
\item
  \texttt{{[}-p\ projectfile{]}}
\item
  \texttt{{[}-o\ output\_folder{]}} is to write all model output
  variables in the specified output directory
\item
  \texttt{{[}-n\ Num\_Threads{]}} is number of OpenMP threads, which
  works with \texttt{pihm\_omp} only.
\end{itemize}

When the \texttt{pihm++} program starts to run, the screen should look
like this: \includegraphics{Fig/CLI_SH.png}

\section{PIHMgisR}\label{pihmgisr}

This PIHMgisR is an R package. What you need is to install the package
as a source code package. For example:

\begin{verbatim}
install_github('shulele/PIHMgisR')
\end{verbatim}

That is all you need to deploy the PIHMgisR.

\chapter{Input and output}\label{input-and-output}

List of input files:

\begin{longtable}[]{@{}ccccc@{}}
\toprule
File & Category & Comments & Header & \# of column\tabularnewline
\midrule
\endhead
.mesh & sp & Domain \textbf{element} (triangular mesh) & Yes
&\tabularnewline
.att & sp & Attribute table of triangular \textbf{elements} & Yes
&\tabularnewline
.riv & sp & \textbf{Rivers} & Yes &\tabularnewline
.rivchn & sp & Topologic relation b/w \textbf{River} and
\textbf{Element} & Yes &\tabularnewline
.calib & cfg & Calibration on physical parameters & Yes &\tabularnewline
.para & cfg & Parameters of the model configurature & Yes
&\tabularnewline
.ic & cfg & Intial conditions & Yes &\tabularnewline
.geol & para & Physical parameters for \textbf{Geology} layers & Yes
&\tabularnewline
.soil & para & Physical parameters for \textbf{Soil} layers & Yes
&\tabularnewline
.lc & para & Physical parameters for \textbf{Land cover} layers & Yes
&\tabularnewline
.forc & tsd & List of files to the Time-series forcing data & Yes
&\tabularnewline
.csv & tsd & Time-series \textbf{forcing} data & Yes &\tabularnewline
.lai & tsd & Time-series \textbf{LAI} data & Yes &\tabularnewline
.obs & tsd & Time-series observational data for calibration purpose only
& Yes &\tabularnewline
.mf & tsd & Time-series \textbf{Melt Factor} data & Yes &\tabularnewline
.rl & tsd & Time-series \textbf{Roughness Length} data & Yes
&\tabularnewline
gis/domain & Shapefile & Shapefile of .mesh file & x & x\tabularnewline
gis/river & Shapefile & Shapefile of .riv file & x & x\tabularnewline
gis/seg & Shapefile & Shapefile of .rivchn file & x & x\tabularnewline
\bottomrule
\end{longtable}

\begin{figure}
\centering
\includegraphics{Fig/IO/Inputfiles.png}
\caption{The screenshot of input files for PIHM++}
\end{figure}

The files in folder \emph{gis} and \emph{fig} are not involved in PIHM
modeling, but they are very useful for your data pre- and
post-processing.

\section{Spatial data}\label{spatial-data}

\subsection{.sp.mesh file}\label{sp.mesh-file}

\includegraphics{Fig/IO/sp.mesh1.png}
\includegraphics{Fig/IO/sp.mesh2.png} There are two tables in the .mesh
file, the one is a table of elements and the other is a table of nodes
of elements.

\begin{itemize}
\item
  \textbf{Block 1 (Element information)}
\item
  Pre-table
\end{itemize}

\begin{longtable}[]{@{}cc@{}}
\toprule
Value1 & Value2\tabularnewline
\midrule
\endhead
Number of rows ( \(N_{element}\)) & Number of columns
(\(8\))\tabularnewline
\bottomrule
\end{longtable}

\begin{itemize}
\tightlist
\item
  Table
\end{itemize}

\begin{longtable}[]{@{}ccccc@{}}
\toprule
Colname & Meaning & Range & Unit & Comments\tabularnewline
\midrule
\endhead
ID & Index of element \(i\) & 1 \textasciitilde{} \(N_{element}\) & -
&\tabularnewline
Node1 & Node 1 of element \(i\) & 1 \textasciitilde{} \(N_{node}\) & -
&\tabularnewline
Node2 & Node 2 of element \(i\) & 1 \textasciitilde{} \(N_{node}\) & -
&\tabularnewline
Node3 & Node 3 of element \(i\) & 1 \textasciitilde{} \(N_{node}\) & -
&\tabularnewline
Nabr1 & Index of Neighbor 1 of element \(i\) & 1 \textasciitilde{}
\(N_{element}\) & - &\tabularnewline
Nabr2 & Index of Neighbor 2 of element \(i\) & 1 \textasciitilde{}
\(N_{element}\) & - &\tabularnewline
Nabr3 & Index of Neighbor 3 of element \(i\) & 1 \textasciitilde{}
\(N_{element}\) & - &\tabularnewline
Zmax & Surface elevation of element \(i\) & -9999 \textasciitilde{} +inf
& \(m\) &\tabularnewline
\bottomrule
\end{longtable}

\begin{itemize}
\item
  \textbf{Block 2 (node information)}
\item
  Pre-table:
\end{itemize}

\begin{longtable}[]{@{}cc@{}}
\toprule
Value1 & Value2\tabularnewline
\midrule
\endhead
Number of rows ( \(N_{node}\)) & Number of columns
(\(5\))\tabularnewline
\bottomrule
\end{longtable}

\begin{itemize}
\tightlist
\item
  Table
\end{itemize}

\begin{longtable}[]{@{}ccccc@{}}
\toprule
Colname & Meaning & Range & Unit & Comments\tabularnewline
\midrule
\endhead
ID & Index of node \(i\) & 1 \textasciitilde{} \(N_{element}\) & -
&\tabularnewline
X & X coordinate of node \(i\) & 1 \textasciitilde{} \(N_{node}\) & -
&\tabularnewline
Y & Y coordinate of node \(i\) & 1 \textasciitilde{} \(N_{node}\) & -
&\tabularnewline
AqDepth & Thickness of aquifer \(i\) & 0 \textasciitilde{} +inf & \(m\)
&\tabularnewline
Elevation & Surface elevation of node \(i\) & -9999 \textasciitilde{}
+inf & \(m\) &\tabularnewline
\bottomrule
\end{longtable}

\subsection{.sp.att file}\label{sp.att-file}

\begin{figure}
\centering
\includegraphics{Fig/IO/sp.att.png}
\caption{Example of .sp.att file}
\end{figure}

\begin{itemize}
\tightlist
\item
  Pre-table
\end{itemize}

\begin{longtable}[]{@{}cc@{}}
\toprule
Value1 & Value2\tabularnewline
\midrule
\endhead
Number of rows ( \(N_{element}\)) & Number of columns
(\(7\))\tabularnewline
\bottomrule
\end{longtable}

\begin{itemize}
\tightlist
\item
  Table
\end{itemize}

\begin{longtable}[]{@{}ccccc@{}}
\toprule
Colname & Meaning & Range & Unit & Comments\tabularnewline
\midrule
\endhead
ID & Index of element \(i\) & 1 \textasciitilde{} \(N_{element}\) & -
&\tabularnewline
SOIL & Index of soil type & 1 \textasciitilde{} \(N_{soil}\) & -
&\tabularnewline
GEOL & Index of geology type & 1 \textasciitilde{} \(N_{geol}\) & -
&\tabularnewline
LC & Index of land cover type & 1 \textasciitilde{} \(N_{lc}\) & - &
\(N_{lc}\) = \(N_{lai}\)\tabularnewline
FORC & Index of forcing site & 1 \textasciitilde{} \(N_{forc}\) & -
&\tabularnewline
MF & Index of melt factor & 1 \textasciitilde{} \(N_{mf}\) & -
&\tabularnewline
BC & Index of boundary condition & 1 \textasciitilde{} \(N_{bc}\) & -
&\tabularnewline
\bottomrule
\end{longtable}

\subsection{.sp.riv file}\label{sp.riv-file}

\begin{figure}
\centering
\includegraphics{Fig/IO/sp.riv.png}
\caption{Example of .sp.riv file}
\end{figure}

\begin{itemize}
\tightlist
\item
  Pre-table
\end{itemize}

\begin{longtable}[]{@{}cc@{}}
\toprule
Value1 & Value2\tabularnewline
\midrule
\endhead
Number of rows ( \(N_{riv}\)) & Number of columns (\(5\))\tabularnewline
\bottomrule
\end{longtable}

\begin{itemize}
\tightlist
\item
  Table
\end{itemize}

\begin{longtable}[]{@{}ccccc@{}}
\toprule
\begin{minipage}[b]{0.11\columnwidth}\centering\strut
Colname\strut
\end{minipage} & \begin{minipage}[b]{0.25\columnwidth}\centering\strut
Meaning\strut
\end{minipage} & \begin{minipage}[b]{0.11\columnwidth}\centering\strut
Range\strut
\end{minipage} & \begin{minipage}[b]{0.11\columnwidth}\centering\strut
Unit\strut
\end{minipage} & \begin{minipage}[b]{0.29\columnwidth}\centering\strut
Comments\strut
\end{minipage}\tabularnewline
\midrule
\endhead
\begin{minipage}[t]{0.11\columnwidth}\centering\strut
ID\strut
\end{minipage} & \begin{minipage}[t]{0.25\columnwidth}\centering\strut
Index of river \(i\)\strut
\end{minipage} & \begin{minipage}[t]{0.11\columnwidth}\centering\strut
1 \textasciitilde{} \(N_{river}\)\strut
\end{minipage} & \begin{minipage}[t]{0.11\columnwidth}\centering\strut
-\strut
\end{minipage} & \begin{minipage}[t]{0.29\columnwidth}\centering\strut
\strut
\end{minipage}\tabularnewline
\begin{minipage}[t]{0.11\columnwidth}\centering\strut
DOWN\strut
\end{minipage} & \begin{minipage}[t]{0.25\columnwidth}\centering\strut
Index of downstream river\strut
\end{minipage} & \begin{minipage}[t]{0.11\columnwidth}\centering\strut
1 \textasciitilde{} \(N_{river}\)\strut
\end{minipage} & \begin{minipage}[t]{0.11\columnwidth}\centering\strut
-\strut
\end{minipage} & \begin{minipage}[t]{0.29\columnwidth}\centering\strut
Negative vlaue indicates outlet\strut
\end{minipage}\tabularnewline
\begin{minipage}[t]{0.11\columnwidth}\centering\strut
Type\strut
\end{minipage} & \begin{minipage}[t]{0.25\columnwidth}\centering\strut
Index of river parameters\strut
\end{minipage} & \begin{minipage}[t]{0.11\columnwidth}\centering\strut
1 \textasciitilde{} \(N_{rivertype}\)\strut
\end{minipage} & \begin{minipage}[t]{0.11\columnwidth}\centering\strut
-\strut
\end{minipage} & \begin{minipage}[t]{0.29\columnwidth}\centering\strut
\strut
\end{minipage}\tabularnewline
\begin{minipage}[t]{0.11\columnwidth}\centering\strut
Slope\strut
\end{minipage} & \begin{minipage}[t]{0.25\columnwidth}\centering\strut
Slope of river bed\strut
\end{minipage} & \begin{minipage}[t]{0.11\columnwidth}\centering\strut
-10 \textasciitilde{} 10\strut
\end{minipage} & \begin{minipage}[t]{0.11\columnwidth}\centering\strut
\(m/m\)\strut
\end{minipage} & \begin{minipage}[t]{0.29\columnwidth}\centering\strut
Height/Length\strut
\end{minipage}\tabularnewline
\begin{minipage}[t]{0.11\columnwidth}\centering\strut
Length\strut
\end{minipage} & \begin{minipage}[t]{0.25\columnwidth}\centering\strut
Length of the river \(i\)\strut
\end{minipage} & \begin{minipage}[t]{0.11\columnwidth}\centering\strut
0 \textasciitilde{} inf\strut
\end{minipage} & \begin{minipage}[t]{0.11\columnwidth}\centering\strut
\(m\)\strut
\end{minipage} & \begin{minipage}[t]{0.29\columnwidth}\centering\strut
\strut
\end{minipage}\tabularnewline
\bottomrule
\end{longtable}

\subsection{.sp.rivchn file}\label{sp.rivchn-file}

\begin{figure}
\centering
\includegraphics{Fig/IO/sp.rivchn.png}
\caption{Example of .sp.rivchn file}
\end{figure}

\begin{itemize}
\tightlist
\item
  Pre-table
\end{itemize}

\begin{longtable}[]{@{}cc@{}}
\toprule
Value1 & Value2\tabularnewline
\midrule
\endhead
Number of rows ( \(N_{segment}\)) & Number of columns
(\(4\))\tabularnewline
\bottomrule
\end{longtable}

\begin{itemize}
\tightlist
\item
  Table
\end{itemize}

\begin{longtable}[]{@{}ccccc@{}}
\toprule
Colname & Meaning & Range & Unit & Comments\tabularnewline
\midrule
\endhead
ID & Index of segments \(i\) & 1 \textasciitilde{} \(N_{segment}\) & -
&\tabularnewline
iRiv & Index of river & 1 \textasciitilde{} \(N_{river}\) & -
&\tabularnewline
iEle & Index of element & 1 \textasciitilde{} \(N_{element}\) & -
&\tabularnewline
Length & Length of the segments \(i\) & 0 \textasciitilde{} inf & \(m\)
&\tabularnewline
\bottomrule
\end{longtable}

\section{Model configuration files}\label{model-configuration-files}

\subsection{.cfg.para file}\label{cfg.para-file}

\begin{figure}
\centering
\includegraphics{Fig/IO/cfg.para.png}
\caption{Example of .cfg.para file}
\end{figure}

\begin{itemize}
\tightlist
\item
  Table
\end{itemize}

\begin{longtable}[]{@{}ccccc@{}}
\toprule
\begin{minipage}[b]{0.17\columnwidth}\centering\strut
Colname\strut
\end{minipage} & \begin{minipage}[b]{0.23\columnwidth}\centering\strut
Meaning\strut
\end{minipage} & \begin{minipage}[b]{0.10\columnwidth}\centering\strut
Range\strut
\end{minipage} & \begin{minipage}[b]{0.10\columnwidth}\centering\strut
Unit\strut
\end{minipage} & \begin{minipage}[b]{0.26\columnwidth}\centering\strut
Comments\strut
\end{minipage}\tabularnewline
\midrule
\endhead
\begin{minipage}[t]{0.17\columnwidth}\centering\strut
VERBOSE\strut
\end{minipage} & \begin{minipage}[t]{0.23\columnwidth}\centering\strut
Verbose mode\strut
\end{minipage} & \begin{minipage}[t]{0.10\columnwidth}\centering\strut
-\strut
\end{minipage} & \begin{minipage}[t]{0.10\columnwidth}\centering\strut
-\strut
\end{minipage} & \begin{minipage}[t]{0.26\columnwidth}\centering\strut
\strut
\end{minipage}\tabularnewline
\begin{minipage}[t]{0.17\columnwidth}\centering\strut
DEBUG\strut
\end{minipage} & \begin{minipage}[t]{0.23\columnwidth}\centering\strut
Debug mode\strut
\end{minipage} & \begin{minipage}[t]{0.10\columnwidth}\centering\strut
-\strut
\end{minipage} & \begin{minipage}[t]{0.10\columnwidth}\centering\strut
-\strut
\end{minipage} & \begin{minipage}[t]{0.26\columnwidth}\centering\strut
\strut
\end{minipage}\tabularnewline
\begin{minipage}[t]{0.17\columnwidth}\centering\strut
INIT\_MODE\strut
\end{minipage} & \begin{minipage}[t]{0.23\columnwidth}\centering\strut
Initial condition mode\strut
\end{minipage} & \begin{minipage}[t]{0.10\columnwidth}\centering\strut
1,2,3\strut
\end{minipage} & \begin{minipage}[t]{0.10\columnwidth}\centering\strut
-\strut
\end{minipage} & \begin{minipage}[t]{0.26\columnwidth}\centering\strut
1=Dry condition, 2=Relief conditon, 3=Warm start\strut
\end{minipage}\tabularnewline
\begin{minipage}[t]{0.17\columnwidth}\centering\strut
ASCII\_OUTPUT\strut
\end{minipage} & \begin{minipage}[t]{0.23\columnwidth}\centering\strut
ASCII ouput\strut
\end{minipage} & \begin{minipage}[t]{0.10\columnwidth}\centering\strut
1/0\strut
\end{minipage} & \begin{minipage}[t]{0.10\columnwidth}\centering\strut
-\strut
\end{minipage} & \begin{minipage}[t]{0.26\columnwidth}\centering\strut
\strut
\end{minipage}\tabularnewline
\begin{minipage}[t]{0.17\columnwidth}\centering\strut
Binary\_OUTPUT\strut
\end{minipage} & \begin{minipage}[t]{0.23\columnwidth}\centering\strut
Binary output\strut
\end{minipage} & \begin{minipage}[t]{0.10\columnwidth}\centering\strut
1/0\strut
\end{minipage} & \begin{minipage}[t]{0.10\columnwidth}\centering\strut
-\strut
\end{minipage} & \begin{minipage}[t]{0.26\columnwidth}\centering\strut
\strut
\end{minipage}\tabularnewline
\begin{minipage}[t]{0.17\columnwidth}\centering\strut
NUM\_OPENMP\strut
\end{minipage} & \begin{minipage}[t]{0.23\columnwidth}\centering\strut
Number of threads for OpenMP\strut
\end{minipage} & \begin{minipage}[t]{0.10\columnwidth}\centering\strut
0 \textasciitilde{} \(N_{threads}\)\strut
\end{minipage} & \begin{minipage}[t]{0.10\columnwidth}\centering\strut
-\strut
\end{minipage} & \begin{minipage}[t]{0.26\columnwidth}\centering\strut
\strut
\end{minipage}\tabularnewline
\begin{minipage}[t]{0.17\columnwidth}\centering\strut
ABSTOL\strut
\end{minipage} & \begin{minipage}[t]{0.23\columnwidth}\centering\strut
Abosolute tolerance for CVODE solver\strut
\end{minipage} & \begin{minipage}[t]{0.10\columnwidth}\centering\strut
1e-6 \textasciitilde{} 0.1\strut
\end{minipage} & \begin{minipage}[t]{0.10\columnwidth}\centering\strut
-\strut
\end{minipage} & \begin{minipage}[t]{0.26\columnwidth}\centering\strut
\strut
\end{minipage}\tabularnewline
\begin{minipage}[t]{0.17\columnwidth}\centering\strut
RELTOL\strut
\end{minipage} & \begin{minipage}[t]{0.23\columnwidth}\centering\strut
Relative tolerance for CVODE solver\strut
\end{minipage} & \begin{minipage}[t]{0.10\columnwidth}\centering\strut
1e-6 \textasciitilde{} 0.1\strut
\end{minipage} & \begin{minipage}[t]{0.10\columnwidth}\centering\strut
-\strut
\end{minipage} & \begin{minipage}[t]{0.26\columnwidth}\centering\strut
\strut
\end{minipage}\tabularnewline
\begin{minipage}[t]{0.17\columnwidth}\centering\strut
INIT\_SOLVER\_STEP\strut
\end{minipage} & \begin{minipage}[t]{0.23\columnwidth}\centering\strut
Initial time step for CVODE solver\strut
\end{minipage} & \begin{minipage}[t]{0.10\columnwidth}\centering\strut
?\strut
\end{minipage} & \begin{minipage}[t]{0.10\columnwidth}\centering\strut
-\strut
\end{minipage} & \begin{minipage}[t]{0.26\columnwidth}\centering\strut
\strut
\end{minipage}\tabularnewline
\begin{minipage}[t]{0.17\columnwidth}\centering\strut
MAX\_SOLVER\_STEP\strut
\end{minipage} & \begin{minipage}[t]{0.23\columnwidth}\centering\strut
Maximum time step for CVODE solver\strut
\end{minipage} & \begin{minipage}[t]{0.10\columnwidth}\centering\strut
?\strut
\end{minipage} & \begin{minipage}[t]{0.10\columnwidth}\centering\strut
-\strut
\end{minipage} & \begin{minipage}[t]{0.26\columnwidth}\centering\strut
\strut
\end{minipage}\tabularnewline
\begin{minipage}[t]{0.17\columnwidth}\centering\strut
LSM\_STEP\strut
\end{minipage} & \begin{minipage}[t]{0.23\columnwidth}\centering\strut
Time step of Evapotranspiration\strut
\end{minipage} & \begin{minipage}[t]{0.10\columnwidth}\centering\strut
1\textasciitilde{}360\strut
\end{minipage} & \begin{minipage}[t]{0.10\columnwidth}\centering\strut
\(min\)\strut
\end{minipage} & \begin{minipage}[t]{0.26\columnwidth}\centering\strut
\strut
\end{minipage}\tabularnewline
\begin{minipage}[t]{0.17\columnwidth}\centering\strut
START\strut
\end{minipage} & \begin{minipage}[t]{0.23\columnwidth}\centering\strut
Start Time\strut
\end{minipage} & \begin{minipage}[t]{0.10\columnwidth}\centering\strut
-\strut
\end{minipage} & \begin{minipage}[t]{0.10\columnwidth}\centering\strut
\(day\)\strut
\end{minipage} & \begin{minipage}[t]{0.26\columnwidth}\centering\strut
\strut
\end{minipage}\tabularnewline
\begin{minipage}[t]{0.17\columnwidth}\centering\strut
END\strut
\end{minipage} & \begin{minipage}[t]{0.23\columnwidth}\centering\strut
End Time\strut
\end{minipage} & \begin{minipage}[t]{0.10\columnwidth}\centering\strut
-\strut
\end{minipage} & \begin{minipage}[t]{0.10\columnwidth}\centering\strut
\(day\)\strut
\end{minipage} & \begin{minipage}[t]{0.26\columnwidth}\centering\strut
\strut
\end{minipage}\tabularnewline
\begin{minipage}[t]{0.17\columnwidth}\centering\strut
STEPSIZE\_FACTOR\strut
\end{minipage} & \begin{minipage}[t]{0.23\columnwidth}\centering\strut
Step size factor\strut
\end{minipage} & \begin{minipage}[t]{0.10\columnwidth}\centering\strut
-\strut
\end{minipage} & \begin{minipage}[t]{0.10\columnwidth}\centering\strut
-\strut
\end{minipage} & \begin{minipage}[t]{0.26\columnwidth}\centering\strut
Temporary value\strut
\end{minipage}\tabularnewline
\begin{minipage}[t]{0.17\columnwidth}\centering\strut
MODEL\_STEPSIZE\strut
\end{minipage} & \begin{minipage}[t]{0.23\columnwidth}\centering\strut
Model step size\strut
\end{minipage} & \begin{minipage}[t]{0.10\columnwidth}\centering\strut
-\strut
\end{minipage} & \begin{minipage}[t]{0.10\columnwidth}\centering\strut
\(min\)\strut
\end{minipage} & \begin{minipage}[t]{0.26\columnwidth}\centering\strut
\strut
\end{minipage}\tabularnewline
\begin{minipage}[t]{0.17\columnwidth}\centering\strut
dt\_ye\_snow\strut
\end{minipage} & \begin{minipage}[t]{0.23\columnwidth}\centering\strut
Time step of output snow storage\strut
\end{minipage} & \begin{minipage}[t]{0.10\columnwidth}\centering\strut
0 \textasciitilde{} inf\strut
\end{minipage} & \begin{minipage}[t]{0.10\columnwidth}\centering\strut
\(m\)\strut
\end{minipage} & \begin{minipage}[t]{0.26\columnwidth}\centering\strut
\strut
\end{minipage}\tabularnewline
\begin{minipage}[t]{0.17\columnwidth}\centering\strut
dt\_ye\_surf\strut
\end{minipage} & \begin{minipage}[t]{0.23\columnwidth}\centering\strut
Time step of output surface storage\strut
\end{minipage} & \begin{minipage}[t]{0.10\columnwidth}\centering\strut
0 \textasciitilde{} inf\strut
\end{minipage} & \begin{minipage}[t]{0.10\columnwidth}\centering\strut
\(m\)\strut
\end{minipage} & \begin{minipage}[t]{0.26\columnwidth}\centering\strut
\strut
\end{minipage}\tabularnewline
\begin{minipage}[t]{0.17\columnwidth}\centering\strut
dt\_ye\_unsat\strut
\end{minipage} & \begin{minipage}[t]{0.23\columnwidth}\centering\strut
Time step of output unsaturated storage\strut
\end{minipage} & \begin{minipage}[t]{0.10\columnwidth}\centering\strut
0 \textasciitilde{} inf\strut
\end{minipage} & \begin{minipage}[t]{0.10\columnwidth}\centering\strut
\(m\)\strut
\end{minipage} & \begin{minipage}[t]{0.26\columnwidth}\centering\strut
\strut
\end{minipage}\tabularnewline
\begin{minipage}[t]{0.17\columnwidth}\centering\strut
dt\_ye\_gw\strut
\end{minipage} & \begin{minipage}[t]{0.23\columnwidth}\centering\strut
Time step of output groundwater head\strut
\end{minipage} & \begin{minipage}[t]{0.10\columnwidth}\centering\strut
0 \textasciitilde{} inf\strut
\end{minipage} & \begin{minipage}[t]{0.10\columnwidth}\centering\strut
\(m\)\strut
\end{minipage} & \begin{minipage}[t]{0.26\columnwidth}\centering\strut
\strut
\end{minipage}\tabularnewline
\begin{minipage}[t]{0.17\columnwidth}\centering\strut
dt\_Qe\_surf\strut
\end{minipage} & \begin{minipage}[t]{0.23\columnwidth}\centering\strut
Time step of output surface element flux\strut
\end{minipage} & \begin{minipage}[t]{0.10\columnwidth}\centering\strut
0 \textasciitilde{} inf\strut
\end{minipage} & \begin{minipage}[t]{0.10\columnwidth}\centering\strut
\(m^3/day\)\strut
\end{minipage} & \begin{minipage}[t]{0.26\columnwidth}\centering\strut
\strut
\end{minipage}\tabularnewline
\begin{minipage}[t]{0.17\columnwidth}\centering\strut
dt\_Qe\_sub\strut
\end{minipage} & \begin{minipage}[t]{0.23\columnwidth}\centering\strut
Time step of output subsurface element flux\strut
\end{minipage} & \begin{minipage}[t]{0.10\columnwidth}\centering\strut
0 \textasciitilde{} inf\strut
\end{minipage} & \begin{minipage}[t]{0.10\columnwidth}\centering\strut
\(m^3/day\)\strut
\end{minipage} & \begin{minipage}[t]{0.26\columnwidth}\centering\strut
\strut
\end{minipage}\tabularnewline
\begin{minipage}[t]{0.17\columnwidth}\centering\strut
dt\_qe\_et0\strut
\end{minipage} & \begin{minipage}[t]{0.23\columnwidth}\centering\strut
Time step of output element flux, interception\strut
\end{minipage} & \begin{minipage}[t]{0.10\columnwidth}\centering\strut
0 \textasciitilde{} inf\strut
\end{minipage} & \begin{minipage}[t]{0.10\columnwidth}\centering\strut
\(m/day\)\strut
\end{minipage} & \begin{minipage}[t]{0.26\columnwidth}\centering\strut
\strut
\end{minipage}\tabularnewline
\begin{minipage}[t]{0.17\columnwidth}\centering\strut
dt\_qe\_et1\strut
\end{minipage} & \begin{minipage}[t]{0.23\columnwidth}\centering\strut
Time step of output element flux, transpiration\strut
\end{minipage} & \begin{minipage}[t]{0.10\columnwidth}\centering\strut
0 \textasciitilde{} inf\strut
\end{minipage} & \begin{minipage}[t]{0.10\columnwidth}\centering\strut
\(m/day\)\strut
\end{minipage} & \begin{minipage}[t]{0.26\columnwidth}\centering\strut
\strut
\end{minipage}\tabularnewline
\begin{minipage}[t]{0.17\columnwidth}\centering\strut
dt\_qe\_et2\strut
\end{minipage} & \begin{minipage}[t]{0.23\columnwidth}\centering\strut
Time step of output element flux, evaporation\strut
\end{minipage} & \begin{minipage}[t]{0.10\columnwidth}\centering\strut
0 \textasciitilde{} inf\strut
\end{minipage} & \begin{minipage}[t]{0.10\columnwidth}\centering\strut
\(m/day\)\strut
\end{minipage} & \begin{minipage}[t]{0.26\columnwidth}\centering\strut
\strut
\end{minipage}\tabularnewline
\begin{minipage}[t]{0.17\columnwidth}\centering\strut
dt\_qe\_etp\strut
\end{minipage} & \begin{minipage}[t]{0.23\columnwidth}\centering\strut
Time step of output element flux, potential ET\strut
\end{minipage} & \begin{minipage}[t]{0.10\columnwidth}\centering\strut
0 \textasciitilde{} inf\strut
\end{minipage} & \begin{minipage}[t]{0.10\columnwidth}\centering\strut
\(m/day\)\strut
\end{minipage} & \begin{minipage}[t]{0.26\columnwidth}\centering\strut
\strut
\end{minipage}\tabularnewline
\begin{minipage}[t]{0.17\columnwidth}\centering\strut
dt\_qe\_prcp\strut
\end{minipage} & \begin{minipage}[t]{0.23\columnwidth}\centering\strut
Time step of output element flux, interception\strut
\end{minipage} & \begin{minipage}[t]{0.10\columnwidth}\centering\strut
0 \textasciitilde{} inf\strut
\end{minipage} & \begin{minipage}[t]{0.10\columnwidth}\centering\strut
\(m/day\)\strut
\end{minipage} & \begin{minipage}[t]{0.26\columnwidth}\centering\strut
\strut
\end{minipage}\tabularnewline
\begin{minipage}[t]{0.17\columnwidth}\centering\strut
dt\_qe\_infil\strut
\end{minipage} & \begin{minipage}[t]{0.23\columnwidth}\centering\strut
Time step of output element flux, interception\strut
\end{minipage} & \begin{minipage}[t]{0.10\columnwidth}\centering\strut
0 \textasciitilde{} inf\strut
\end{minipage} & \begin{minipage}[t]{0.10\columnwidth}\centering\strut
\(m/day\)\strut
\end{minipage} & \begin{minipage}[t]{0.26\columnwidth}\centering\strut
\strut
\end{minipage}\tabularnewline
\begin{minipage}[t]{0.17\columnwidth}\centering\strut
dt\_qe\_rech\strut
\end{minipage} & \begin{minipage}[t]{0.23\columnwidth}\centering\strut
Time step of output element flux, interception\strut
\end{minipage} & \begin{minipage}[t]{0.10\columnwidth}\centering\strut
0 \textasciitilde{} inf\strut
\end{minipage} & \begin{minipage}[t]{0.10\columnwidth}\centering\strut
\(m/day\)\strut
\end{minipage} & \begin{minipage}[t]{0.26\columnwidth}\centering\strut
\strut
\end{minipage}\tabularnewline
\begin{minipage}[t]{0.17\columnwidth}\centering\strut
dt\_yr\_stage\strut
\end{minipage} & \begin{minipage}[t]{0.23\columnwidth}\centering\strut
Time step of output river stage\strut
\end{minipage} & \begin{minipage}[t]{0.10\columnwidth}\centering\strut
0 \textasciitilde{} inf\strut
\end{minipage} & \begin{minipage}[t]{0.10\columnwidth}\centering\strut
\(m^3/day\)\strut
\end{minipage} & \begin{minipage}[t]{0.26\columnwidth}\centering\strut
\strut
\end{minipage}\tabularnewline
\begin{minipage}[t]{0.17\columnwidth}\centering\strut
dt\_Qr\_down\strut
\end{minipage} & \begin{minipage}[t]{0.23\columnwidth}\centering\strut
Time step of output river flux, downstream\strut
\end{minipage} & \begin{minipage}[t]{0.10\columnwidth}\centering\strut
0 \textasciitilde{} inf\strut
\end{minipage} & \begin{minipage}[t]{0.10\columnwidth}\centering\strut
\(m^3/day\)\strut
\end{minipage} & \begin{minipage}[t]{0.26\columnwidth}\centering\strut
\strut
\end{minipage}\tabularnewline
\begin{minipage}[t]{0.17\columnwidth}\centering\strut
dt\_Qr\_surf\strut
\end{minipage} & \begin{minipage}[t]{0.23\columnwidth}\centering\strut
Time step of output river flux, surface flow\strut
\end{minipage} & \begin{minipage}[t]{0.10\columnwidth}\centering\strut
0 \textasciitilde{} inf\strut
\end{minipage} & \begin{minipage}[t]{0.10\columnwidth}\centering\strut
\(m^3/day\)\strut
\end{minipage} & \begin{minipage}[t]{0.26\columnwidth}\centering\strut
\strut
\end{minipage}\tabularnewline
\begin{minipage}[t]{0.17\columnwidth}\centering\strut
dt\_Qr\_sub\strut
\end{minipage} & \begin{minipage}[t]{0.23\columnwidth}\centering\strut
Time step of output river flux, base flow\strut
\end{minipage} & \begin{minipage}[t]{0.10\columnwidth}\centering\strut
0 \textasciitilde{} inf\strut
\end{minipage} & \begin{minipage}[t]{0.10\columnwidth}\centering\strut
\(m^3/day\)\strut
\end{minipage} & \begin{minipage}[t]{0.26\columnwidth}\centering\strut
\strut
\end{minipage}\tabularnewline
\begin{minipage}[t]{0.17\columnwidth}\centering\strut
dt\_Qr\_up\strut
\end{minipage} & \begin{minipage}[t]{0.23\columnwidth}\centering\strut
Time step of output river flux, upstream\strut
\end{minipage} & \begin{minipage}[t]{0.10\columnwidth}\centering\strut
0 \textasciitilde{} inf\strut
\end{minipage} & \begin{minipage}[t]{0.10\columnwidth}\centering\strut
\(m^3/day\)\strut
\end{minipage} & \begin{minipage}[t]{0.26\columnwidth}\centering\strut
\strut
\end{minipage}\tabularnewline
\bottomrule
\end{longtable}

\subsection{.cfg.calib file}\label{cfg.calib-file}

\begin{figure}
\centering
\includegraphics{Fig/IO/cfg.calib.png}
\caption{Example of .cfg.calib file}
\end{figure}

\begin{itemize}
\tightlist
\item
  Table
\end{itemize}

\begin{longtable}[]{@{}ccccc@{}}
\toprule
Colname & Meaning & Range & Unit & Comments\tabularnewline
\midrule
\endhead
KSATH & Horizontal conductivity of ground water & ? & - &\tabularnewline
KSATV & Vertical conductivity of ground water & ? & - &\tabularnewline
KINF & Vertical conductivity of top soil & ? & - &\tabularnewline
KMACSATH & Horizontal conductivity of macropore & ? & - &\tabularnewline
KMACSATV & Vertical conductivity of soil macropore & ? & -
&\tabularnewline
DINF & Infiltration depth & ? & - &\tabularnewline
DROOT & Root depth & & - &\tabularnewline
DMAC & Macropore depth & & - &\tabularnewline
THETAS & Porosity, saturated soil moisture & & - &\tabularnewline
THETAR & Residual soil moisture & & - &\tabularnewline
ALPHA & \(\alpha\) value in van Genuchten equation & & -
&\tabularnewline
BETA & \(\beta\) value in van Genuchten equation & & - &\tabularnewline
MACVF & Vertical macropore areal fraction & & - &\tabularnewline
MACHF & Horizontal macropore areal fraction & & - &\tabularnewline
VEGFRAC & Vegetation fraction & & - &\tabularnewline
ALBEDO & Emissitive reflection ratio & & - &\tabularnewline
ROUGH & Manning's roughness of element surface & & - &\tabularnewline
AQUIFER & Thichness of aquifer & & - &\tabularnewline
PRCP & Precipitation & & - &\tabularnewline
SFCTMP & Temperature & & - &\tabularnewline
EC & Interception & & - &\tabularnewline
ETT & Transpiration & & - &\tabularnewline
EDIR & Evaporation & & - &\tabularnewline
RIV\_ROUGH & Manning's roughness of river & & - &\tabularnewline
RIV\_KH & Conductivity of river bed & & - &\tabularnewline
RIV\_DPTH & Depth of river cross section & & - &\tabularnewline
RIV\_WDTH & Width of river cross section & & - &\tabularnewline
RIV\_SINU & Sinuosity of river path & & - &\tabularnewline
RIV\_CWR & \(C_{wr}\) in Chezy equation & & - &\tabularnewline
RIV\_BSLOPE & Slope of river bed & & - &\tabularnewline
SOIL\_DGD & Soil degradation & & - &\tabularnewline
IMPAF & Impervious areal fraction & & - &\tabularnewline
ISMAX & Maximum interception & & - &\tabularnewline
\bottomrule
\end{longtable}

\subsection{.cfg.ic file}\label{cfg.ic-file}

\begin{figure}
\centering
\includegraphics{Fig/IO/cfg.ic.png}
\caption{Example of .cfg.ic file}
\end{figure}

\begin{itemize}
\item
  \textbf{Block 1 (Element initial condition)}
\item
  Pre-table
\end{itemize}

\begin{longtable}[]{@{}cc@{}}
\toprule
Value1 & Value2\tabularnewline
\midrule
\endhead
Number of rows ( \(N_{element}\)) & Number of columns
(\(6\))\tabularnewline
\bottomrule
\end{longtable}

\begin{itemize}
\tightlist
\item
  Table
\end{itemize}

\begin{longtable}[]{@{}ccccc@{}}
\toprule
Colname & Meaning & Range & Unit & Comments\tabularnewline
\midrule
\endhead
ID & Index of element \(i\) & 1 \textasciitilde{} \(N_{element}\) & -
&\tabularnewline
Canopy & Canopy storage of element \(i\) & 0 \textasciitilde{} inf &
\(m\) &\tabularnewline
Snow & Snow storage of element \(i\) & 0 \textasciitilde{} inf & \(m\)
&\tabularnewline
Surface & Surface storage of element \(i\) & 0 \textasciitilde{} inf &
\(m\) &\tabularnewline
Unsat & Unsaturated storage of element \(i\) & 0 \textasciitilde{} inf &
\(m\) &\tabularnewline
GW & Groundwater head of element \(i\) & 0 \textasciitilde{} inf & \(m\)
&\tabularnewline
\bottomrule
\end{longtable}

\begin{itemize}
\item
  \textbf{Block 2 (river initial condition)}
\item
  Pre-table:
\end{itemize}

\begin{longtable}[]{@{}cc@{}}
\toprule
Value1 & Value2\tabularnewline
\midrule
\endhead
Number of rows ( \(N_{riv}\)) & Number of columns (\(2\))\tabularnewline
\bottomrule
\end{longtable}

\begin{itemize}
\tightlist
\item
  Table
\end{itemize}

\begin{longtable}[]{@{}ccccc@{}}
\toprule
Colname & Meaning & Range & Unit & Comments\tabularnewline
\midrule
\endhead
ID & Index of river \(i\) & 1 \textasciitilde{} \(N_{riv}\) & -
&\tabularnewline
Stage & Stage of river \(i\) & 0 \textasciitilde{} inf & \(m\)
&\tabularnewline
\bottomrule
\end{longtable}

\section{Time-series data}\label{time-series-data}

\subsection{.tsd.forc file}\label{tsd.forc-file}

\begin{figure}
\centering
\includegraphics{Fig/IO/tsd.forc.png}
\caption{Example of .tsd.forc file}
\end{figure}

\begin{itemize}
\tightlist
\item
  Line 1:
  \texttt{Number\ of\ forcing\ sites\ \textbar{}\ Start\ day\ (YYYYMMDD)}
\item
  Line 2: Directory to the spreadsheet
\item
  Line 3\textasciitilde{}N: Filenames of spreadsheet
\end{itemize}

\subsection{.tsd.lai file}\label{tsd.lai-file}

\begin{figure}
\centering
\includegraphics{Fig/IO/tsd.lai.png}
\caption{Example of .tsd.lai file}
\end{figure}

\begin{itemize}
\tightlist
\item
  Pre-table:
\end{itemize}

\begin{longtable}[]{@{}ccc@{}}
\toprule
Value1 & Value2 & Value3\tabularnewline
\midrule
\endhead
Number of day ( \(N_{time}\)) & Number of columns (\(N_{lc}\)) & Start
day (YYYYMMDD)\tabularnewline
\bottomrule
\end{longtable}

\begin{itemize}
\tightlist
\item
  Table
\end{itemize}

\begin{longtable}[]{@{}ccccc@{}}
\toprule
Colname & Meaning & Range & Unit & Comments\tabularnewline
\midrule
\endhead
TIME & Time & 0 \textasciitilde{} \(N_{time}\) & \(day\)
&\tabularnewline
Column 2 & LAI of land cover 1 & 0 \textasciitilde{} inf & \(m^2/m^2\)
&\tabularnewline
Column i & LAI of land cover \(i-1\) & 0 \textasciitilde{} inf &
\(m^2/m^2\) &\tabularnewline
\ldots{} & \ldots{} & \ldots{} & \ldots{} &\tabularnewline
\bottomrule
\end{longtable}

\subsection{.tsd.rl file}\label{tsd.rl-file}

\begin{figure}
\centering
\includegraphics{Fig/IO/tsd.rl.png}
\caption{Example of .tsd.rl file}
\end{figure}

\begin{itemize}
\tightlist
\item
  Pre-table:
\end{itemize}

\begin{longtable}[]{@{}ccc@{}}
\toprule
Value1 & Value2 & Value3\tabularnewline
\midrule
\endhead
Number of day ( \(N_{time}\)) & Number of columns (\(N_{lc}\)) & Start
day (YYYYMMDD)\tabularnewline
\bottomrule
\end{longtable}

\begin{itemize}
\tightlist
\item
  Table
\end{itemize}

\begin{longtable}[]{@{}ccccc@{}}
\toprule
Colname & Meaning & Range & Unit & Comments\tabularnewline
\midrule
\endhead
TIME & Time & 0 \textasciitilde{} \(N_{time}\) & \(day\)
&\tabularnewline
Column 2 & Roughness length of land cover 1 & 0 \textasciitilde{} inf &
\(m\) &\tabularnewline
Column i & Roughness length of land cover \(i-1\) & 0 \textasciitilde{}
inf & \(m\) &\tabularnewline
\ldots{} & \ldots{} & \ldots{} & \ldots{} &\tabularnewline
\bottomrule
\end{longtable}

\subsection{.tsd.mf file}\label{tsd.mf-file}

\begin{figure}
\centering
\includegraphics{Fig/IO/tsd.mf.png}
\caption{Example of .tsd.mf file}
\end{figure}

\begin{itemize}
\tightlist
\item
  Pre-table:
\end{itemize}

\begin{longtable}[]{@{}ccc@{}}
\toprule
Value1 & Value2 & Value3\tabularnewline
\midrule
\endhead
Number of day ( \(N_{time}\)) & Number of columns (\(N_{mf}\)) & Start
day (YYYYMMDD)\tabularnewline
\bottomrule
\end{longtable}

\begin{itemize}
\tightlist
\item
  Table
\end{itemize}

\begin{longtable}[]{@{}ccccc@{}}
\toprule
Colname & Meaning & Range & Unit & Comments\tabularnewline
\midrule
\endhead
TIME & Time & 0 \textasciitilde{} \(N_{time}\) & \(day\)
&\tabularnewline
Column 2 & Melt factor 1 & 0 \textasciitilde{} inf & - &\tabularnewline
Column i & Melt factor \(i-1\) & 0 \textasciitilde{} inf & -
&\tabularnewline
\ldots{} & \ldots{} & \ldots{} & \ldots{} &\tabularnewline
\bottomrule
\end{longtable}

\subsection{.tsd.obs file}\label{tsd.obs-file}

\begin{figure}
\centering
\includegraphics{Fig/IO/tsd.obs.png}
\caption{Example of .tsd.obs file}
\end{figure}

\begin{itemize}
\tightlist
\item
  Pre-table:
\end{itemize}

\begin{longtable}[]{@{}ccc@{}}
\toprule
Value1 & Value2 & Value3\tabularnewline
\midrule
\endhead
Number of day ( \(N_{time}\)) & Number of columns (\(N_{obs}\)) & Start
day (YYYYMMDD)\tabularnewline
\bottomrule
\end{longtable}

\begin{itemize}
\tightlist
\item
  Table
\end{itemize}

\begin{longtable}[]{@{}ccccc@{}}
\toprule
Colname & Meaning & Range & Unit & Comments\tabularnewline
\midrule
\endhead
TIME & Time & 0 \textasciitilde{} \(N_{time}\) & \(day\)
&\tabularnewline
Column 2 & Observational data 1 & ? & ? &\tabularnewline
Column i & Observational data \(i-1\) & ? & ? &\tabularnewline
\ldots{} & \ldots{} & \ldots{} & \ldots{} &\tabularnewline
\bottomrule
\end{longtable}

\chapter{Applications}\label{applications}

Some \emph{significant} applications are demonstrated in this chapter.

\section{Example 1: Vauclin
Experiment}\label{example-1-vauclin-experiment}

\section{Example 2: Shall Hill CZO}\label{example-2-shall-hill-czo}

\section{Example 3: Conestoga Watershed,
Pennsylvanis}\label{example-3-conestoga-watershed-pennsylvanis}

\chapter{Automatic hydrologic modeling with PIHM
system}\label{automatic-hydrologic-modeling-with-pihm-system}

Automatic deployment of PIHM System

\chapter{Course code and program
design}\label{course-code-and-program-design}

The source code of PIHM++ and PIHMgisR are avaliable via Github:
\url{https://github.com/shulele/PIHM-4.0} and
\url{https://github.com/shulele/PIHMgisR}.

\bibliography{book.bib}

\end{document}
